\section{Практика}\marginpar{\framebox{10.04.2014}}
\begin{tsk}\label{exs:pr:3}
\begin{eqnarray}
\max x_1+x_2\\
2x_1 + 11x_2 \leq 38 \\
x_1 + x_2 \leq 7\\
4x_1 - 5x_2 \leq 5 \\
x_1,x_2\geq 0\\
x_1,x_2\in\mZ\\
d_1^0 = d_2^0 = 0
\end{eqnarray} 
Знадемо множину допустимих значень для $x_1$ та $x_2$ з обмежень. Вона має бути найширшою.
\begin{eqnarray}
x_1^0 = \set{0,7}\\
x_2^0 = \set{0,3}
\end{eqnarray}
Відсіюємо недопустимі варіанти
\begin{eqnarray}
&\omg_1(x_1):&2x_1 > 38 - \min \set{11x_2}\\
&&x_1 > 7 -\min\set{x_2}\\
&&4x_1 >5 + \max\set{5x_2}\\
&&x_1^0 = \set{0,5}\\
&\omg_1(x_2):&11x_1 > 38 - \min\set{2x_1}\\
&&x_2>7 - \min\set{x_1} \\
&&5x_2 > 5 - \min\set{4x_2}\\
\end{eqnarray}
Переходимо до відсіву значень за значеннями цільової функції
\begin{eqnarray}
&\omg_2:&f_1^\ast = \cfrac12 \cb{\min f(x) +\max f(x)}\\
&& f_1^\ast = \cfrac12\cdot 8 = 4 \\
&& x_1 < f_1^\ast - \max\set{x_2}\\
&& x_1 < 1 \Rightarrow x_1^0 = \set{1,5}\\
&& x_2 < f_1^\ast - \max\set{x_1}\\
&& x_2 < -1 
\end{eqnarray}
Отже, за $x_2$ немає відсіву і йдемо на процедуру $\omg_1$. Зауважимо те, що змінилася лише нижня границя $x_1$, тому має сенс робити цю процедуру лише для $x_2$.
\begin{eqnarray}
&\omg_1(x_2):&x_2 > \cfrac{36}{11} \\
&& x_2 > 6 \\
&& x_2 < -\cfrac14 
\end{eqnarray}
Відсіву не відбулося. Далі робити відсів немає сенсу. \\
Переходимо на процедуру $\omg_2$ і вводимо новий поріг відсіву:
\begin{eqnarray}
&\omg_2:& f_2^\ast = \cfrac12\cb{f_1^\ast +\max f(x)}  = \cfrac12\cb{4+8} = 6\\
&& x_1 < 6 - \max\set{x_2} \\
&& x_1 < 3 \Rightarrow x_1^0 = \set{3,5}\\
&& x_2 < 6 - \max\set{x_1}\\
&& x_2 < 1 \Rightarrow x_2^0 = \set{1,3}
\end{eqnarray}
Аналогічно попереднім роздумам переходимо до процедури $\omg_1$, враховуючи те, що змінилися обидві множини і лише з нижними границями. Тобто, перевіряємо лише там, де мінімум.
\begin{eqnarray}
&\omg_1(x_1):& x_1 > \cfrac{27}2 \\
&& x_1 > 6 \\
&\omg_1(x_2):& x_2 > \cfrac{32}{11} \Rightarrow x_2^0 = \set{1,2}\\
&& x_2 > 4 \\
&& x_2 < \cfrac75 \Rightarrow x_2^0 = 2
\end{eqnarray}
Оскільки множини змінилося, то ми повторюємо процедур $\omg_1$ лише для $x_1$.
\begin{eqnarray}
&\omg_1(x_1):& x_1 > 8\\
&& x_1 > 5\\
&& x_1 > \cfrac{15}4  \Rightarrow x_1^0 = 3
\end{eqnarray}
\end{tsk}

