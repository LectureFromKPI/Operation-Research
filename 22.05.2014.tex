\section{Математичні задачі динамічного програмування} \marginpar{\framebox{22.05.2014}}
\begin{eqnarray}
	&\max \suml_{j=1}^n f_j(x_j)\\
	&\suml_{j=1}^n a_{1i} \xj \leq b_1 \\
	&\suml_{j=1}^n a_{2i} \xj \leq b_2\\
	&x_j \geq 0 \in \mZ
\end{eqnarray}
Так як два обмеження, необхідно ввести два параметра $\xi_1 = \overline{0,b_1};\xi_2 = \overline{0,b_2}$. Задані початкові умови, отже розв’язуємо у прямому напрямку. Запишемо рекурентну формулу:
\begin{equation}
	\Lambda_k \cb{\xi_1,\xi_2} = \max\limits_{x_k\leq \min\set{\bb{\frac{\xi_1}{a_{1k}}};\bb{\frac{\xi_2}{a_{2k}}}} }\set{f_k(x_k) \Lambda_{k-1} \cb{ \xi_1 - a_{1k} x_k;\xi_2 - a_{2k}x_k }}
\end{equation}
Використовуючи це співвідношення визначаємо $\Lambda_1,\ldots,\Lambda_{n-1}$. На останньому кроці $\xi_1 = b_1,\xi_2=b_2$
\begin{equation}
	\Lambda_n\cb{b_1,b_2} = \max\limits_{x_k\leq \min\set{\bb{\frac{b_1}{a_{1k}}};\bb{\frac{b_2}{a_{2k}}}} }  \set{f_n(x_n) \Lambda_{n-1} \cb{b_1 - a_{1n}x_n; b_2 - a_{2n} x_n}}
\end{equation}
З цього співвідношення знаходимо $x_n^o$. Далі, покладемо $\xi_1^o = b_1 - a_{1n}x_n^o;\xi_2^o = b_2 - a_{2n} x_n^o$ і за таблицями останніх кроків визначаємо інші оптимальні.\\
Кількість операцій експоненціально зростає відносно кількості значень. Таке явище Белман назвав \textbf{прокляття розмірності}.
\subsection{Метод множників Лагранджа для зниження розмірності задачі}
Нехай задана наступна задача динамічного програмування 
\begin{eqnarray}
	&\max \suml_{j=1}^n f_j(x_j)\\
	&\suml_{j=1}^n a_{1i} \xj = b_1 \\
	&\suml_{j=1}^n a_{2i} \xj = b_2 \label{tr:11:1}\\
	&x_j \geq 0 \in \mZ
\end{eqnarray}
Введемо друге обмеження (обмеження \eqref{tr:11:1}) в цільову функцію:
\begin{eqnarray}
	&\max \cb{\suml_{j=1}^n f_j(x_j) - \la \suml_{j=1}^n a_{2j} x_j}\\
	&\suml_{j=1}^n a_{1j} x_j = b_1
\end{eqnarray}
Так як у нас лише одне обмеження, то потрібно ввести лише один параметр $\xi = \overline{0,b_1}$. Запишемо рекурентне співвідношення.
\begin{equation}
	\Lambda_k(\xi) = \max \set{f_k(x_k) - \la a_{2k} x_k + \Lambda_{k-1}\cb{\xi - a_{1k} x_k}}
\end{equation}
Використовуючи рекурентне співвідношення визначаємо $x_j^o(\la)$. Так як $\la$ - невідома, задаємо початкове значення, визначаємо $x_j^o$ і перевіряємо, чи виконується обмеження \eqref{tr:11:1}. Якщо воно виконується, то даний розв’язок і буде шуканим. В іншому випадку $\la$ необхідно скоригувати, при цьому, якщо $\sumjon a_{2j}x_j(\la)>b_2$, то $\la$ потрібно збільшити. В іншому випадку - зменшуємо.
\section{Прямі методи одновимірного пошуку}
Запишемо математичну модель:
\begin{eqnarray}
	&\min f(x) \\
	&a\leq x \leq b 
\end{eqnarray}
$f(x)$ - нелінійна функція, $\bb{a,b}$ - інтервал невизначеності.\\
Нехай $\bb{\la,\mu}\subset\bb{a,b}$. Якщо $f(\la)>f(\mu)$, то новий інтервал невизначеності $\bb{\la,b}$. В іншому випадку, новий інтервал невизначеності $\bb{a,\mu}$.
\subsection{Метод Фібоначчі}
\paragraph{Початковий етап} Числа Фібоначчі
\begin{equation*}
	F_0 = F_1 = 1 \quad F_{n+1} = F_n + F_{n-1}
\end{equation*}
Нехай $\bb{a_1,b_1}$ - початковий інтервал невизначеності. Задаємося константою розрізнюваності $\eps>0$ та кінцевою довжиною інтервалу невизначеності $l>0$. Визначаємо кількість ітерацій
\begin{equation}
	n:\quad F_n > \cfrac{b_1 - a_1}{l}
\end{equation}
Визначаємо $\mu_1,\la_1$ за такими формулами:
\begin{eqnarray}
	&\la_1 = a_1 + \cfrac{F_{n-2}}{F_n} \cb{b_1-a_1}\\
	&\mu_1 = a_1 + \cfrac{F_{n-1}}{F_n} \cb{b_1-a_1}
\end{eqnarray}
Визначили $f(\la_1)$ та $f(\mu_1)$ і йдемо на першу ітерацію.
\paragraph{k-та ітерація та перший крок} Порівнюємо $f(\la_k)$ та $f(\mu_k)$. Якщо $f(\la_k)>f(\mu_k)$ йдемо на другий крок, інакше на третій.
\paragraph{Другий крок} 
$a_{k+1} = \la_{k_1}; b_{k+1} = b_{k_1}; \la_{k+1} = \mu_k$, де
\begin{equation}
	\mu_{k+1} = a_{k+1} + \cfrac{F_{n-k-1}}{F_{n-k}} \cb{b_{k+1} - a_{k+1}}
\end{equation}
Якщо $k=n-2$ на четвертий крок, інакше порахували значення функцій і йдемо на наступну ітерацію.
\paragraph{Третій крок}
$a_{k+1} = a_{k}; b_{k+1} = \mu_k; \mu_{k+1} = \la_k$
\begin{equation}
	\la_{k+1} = a_{k+1} + \cfrac{F_{n-k-2}}{F_{n-k}} \cb{b_{k+1} - a_{k+1}}
\end{equation}
Якщо $k=n-2$ на четвертий крок, інакше порахували значення функцій і йдемо на наступну ітерацію.
\paragraph{Четвертий крок}
Покладемо $\la_n = \la_{n-1},\mu_n = \la_n + \eps$. Якщо $f(\la_n)>f(\mu_n)$, то $a_n=\la_n, b_n=b_{n-1}$. В іншому випадку $a_n = a_{n-1};b_n = \mu_n$.\\
Оптимальний розв’язок лежить в інтервалі $\bb{a_n,b_n}$.
\subsection{Дихотомічний пошук}
\paragraph{Початковий етап}
Нехай $\bb{a_1,b_1}$ - інтервал невизначеності, Задаємося константою розрізнюваності $\eps>0$ та кінцевою довжиною інтервалу невизначеності $l>0$.\\
Визначаємо $\la_1,\mu_1$
\begin{eqnarray}
	&\mu_1 = \cfrac{b_1+a_1}{2} + \eps\quad
	&\la_1 = \cfrac{b_1+a_1}{2} - \eps
\end{eqnarray}
Йдемо на першу ітерацію.
\paragraph{k-та ітерація і перший крок}
Якщо $b_k - a_k \leq l$, то оптимальний розв’язок лежить в інтервалі $\bb{a_k,b_k}$. І іншому випадку порівнюємо $f(\la_k)$ та $f(\mu_k)$. Якщо $f(\la_k)>f(\mu_k)$, то другий крок. Інакше - третій.
\paragraph{Другий крок}
$a_{k+1} = \la_k;b_{k+1} = b_k;\la_{k+1} = \mu_k$
\begin{equation}
	\mu_{k+1} = \cfrac{b_{k+1}+a_{k+1}}{2} + \eps
\end{equation}
Визначаємо $f(\la_{k+1});f(\mu_{k+1})$ і йдемо на наступну ітерацію. 
\paragraph{Третій крок}
$a_{k+1} = a_k;b_{k+1} = \mu_k; \mu_{k+1} = \la_k $
\begin{equation}
	\la_{k+1} = \cfrac{b_{k+1}+a_{k+1}}{2} - \eps
\end{equation}
Визначаємо $f(\la_{k+1});f(\mu_{k+1})$ і йдемо на наступну ітерацію.
\subsection{Метод золотого перетину}
\paragraph{Початковий етап}
Нехай $\bb{a_1,b_1}$ - інтервал невизначеності, Задаємося кінцевою довжиною інтервалу невизначеності $l>0$. $\al = 0.618$.\\
Визначаємо $\la_1,\mu_1$
\begin{eqnarray}
	&\mu_1 = a_1 + \al\cb{b_1 - a_1}\quad
	&\la_1 =  a_1 +	cb{1-\al}\cb{b_1 - a_1}
\end{eqnarray}
Йдемо на першу ітерацію.
\paragraph{k-та ітерація і перший крок}
Якщо $b_k - a_k \leq l$, то оптимальний розв’язок лежить в інтервалі $\bb{a_k,b_k}$. І іншому випадку порівнюємо $f(\la_k)$ та $f(\mu_k)$. Якщо $f(\la_k)>f(\mu_k)$, то другий крок. Інакше - третій.
\paragraph{Другий крок}
$a_{k+1} = \la_k;b_{k+1} = b_k;\la_{k+1} = \mu_k$
\begin{equation}
	\mu_{k+1} = a_{k+1} + \al\cb{b_{k+1} - a_{k+1}}
\end{equation}
Визначаємо $f(\la_{k+1});f(\mu_{k+1})$ і йдемо на наступну ітерацію. 
\paragraph{Третій крок}
$a_{k+1} = a_k;b_{k+1} = \mu_k; \mu_{k+1} = \la_k $
\begin{equation}
	\la_{k+1} = a_{k+1} +	cb{1-\al}\cb{b_{k+1} - a_{k+1}}
\end{equation}
Визначаємо $f(\la_{k+1});f(\mu_{k+1})$ і йдемо на наступну ітерацію.
\begin{exs}[Задача керування запасами]
Плануємо роботу на 4 місяця.
$d_1=90,d_2=125,d_3=140,d_4=100,S_i=2$.\\
$A_i=300, y_1 = y_5 = 0$.
Запишемо математичну модель
\begin{eqnarray}
	&\min \cb{300\cdot \suml_{j=1}^4 \delta_j(x_j)} + 2 \suml_{j=1}^4 y_{j+1}\\
	&x_j+y_j-d_j = y_{j+1}
\end{eqnarray}
Де $\delta_j$ - індикатор замовлення.\\
Вигідно розв’язувати її у прямому напрямі) Тому саме так і розв’язуємо.
Перший місяць
\begin{equation}
	\Lambda_1(0) = 300
\end{equation}
Другий місяць:
\begin{equation}
	\Lambda_2(0) = \vect{g_2(1) = A + 2\cdot d_2 = 550 \cb{\ast}\\ g_2(2) = A + \Lambda_1(0) = 600}
\end{equation}
Третій місяць:
\begin{equation}
	\Lambda_3(0) = \vect{g_3(1) = A + 2\cb{d_2+d_3} + 2 \cdot d_3 = 1110  \\ g_3(2) = A + \Lambda_1(0) + 2 \cdot d_3 =880 \\ g_3(3) = A + \Lambda_2(0) = 850 \cb{\ast} \\}
\end{equation}
\end{exs}