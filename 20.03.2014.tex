\begin{exs}\marginpar{\framebox{20.03.2014}}
\label{exam:pr:1}
Задача комвіяжера на шість міст.\\
Необхідно виконати процедуру приведення. Для цього в кожному рядку шукаємо мінімальний (позначимо його $h_i$) і віднімаємо від усіх елементів рядку.
\begin{equation}
c = \begin{array}{c|c|c|c|c|c|c|c}
& 0 & 1 & 2 & 3 & 4 & 5 & h_i \\
\hline
0 & x & 5 & 1 & 10 & 6 & 8 &  1\\
\hline 
1 & 2 & x & 3 & 8 & 5 & 10 & 2 \\
\hline
2 & 4 & 8 & x & 6 & 4 & 9 & 4\\
\hline
3 & 6 & 9 & 5 & x & 3 & 7 & 3\\
\hline
4 & 8 & 7 & 7 & 4 & x & 6 & 4\\
\hline
5 & 10 & 4 & 9 & 3 & 2 & x &2\\
\hline
\end{array}
\end{equation}
Перетворуюємо стовбички, для цього в кожному шукаємо мінімальний (позначаємо його $H_j$) і потім віднімаємо від усіх елементів стовпчику.
\begin{equation}
C' = \begin{matrix}
&x & 4 & 0 & 9 & 5 & 7\\
&0 & x & 1 & 6 & 3 & 8\\
&0 & 4 & x & 2 & 0 & 5\\
&3 & 6 & 2 & x & 0 & 4\\
&4 & 3 & 3 & 0 & x & 2\\
&8 & 2 & 7 & 1 & 0 & x\\
\hline
H_j &0 & 2 & 0 & 0 & 0 & 2
\end{matrix}
\end{equation}
Далі шукаємо оцінки $\al_r$ - це наступний найменший після мінімального у рядку, а $\beta_m$ - наступний найменший після мінімального у стовпчику.
\begin{equation}
C_0 =\left( \begin{array}{c|ccccccc|c}
  & 0 & 1 & 2 & 3 & 4 & 5 & \al_i\\
\hline
0&x & 2 & 0 & 9 & 5 & 5 & 2\\
1&0 & x & 1 & 6 & 3 & 6 & 1\\
2&0 & 2 & x & 2 & 0 & 3 & 0\\
3&3 & 4 & 2 & x & 0 & 2 & 2\\
4&4 & 1 & 4 & 0 & x & 0 & 0\\
5&8 & 0 & 7 & 1 & 0 & x & 0\\
\hline 
\beta_j &0 & 1 & 1 & 1 & 0 & 2 &
\end{array}\right)
\end{equation}
Отримали оцінку: $G_0\cb{\xi} = \suml_{i} h_i + \suml_{j} H_j = 20$. Далі, обчислюємо оцінки для нульових елементів за формулою
\begin{equation}
	\cij:\quad \Theta\cb{\cij} = \al_i + \beta_j
\end{equation}
І розбиваємо по тому елементу, де $\max\Theta$. Це елемент (0,2), $\Theta_{0,2} = 3$.\\
Розбиваємо множину на підмножини:\\
\begin{center}
\begin{tikzpicture}
\node[circle,draw] (c1) {$G_0$};
\node[circle,draw,below left=of c1] (c11) {$G_1$};
\node[circle,draw,below right=of c1] (c12) {$G_2$};
\path[-]
(c1) edge node[above,rotate=45]{(0,2)}(c11)
(c1) edge node[above,rotate=-45]{$\overline{(0,2)}$}(c12)
;
\end{tikzpicture}
\end{center}
Отримали оцінку для другої підмножини $G_2(\xi)=G_0\cb{\xi} + \Theta_{0,2} = 23$. Отже, має йти на $G_1$.\\
Викреслимо другий стовпчик та нульовий рядок (як у переході), забороняємо перехід (2,0) та отримаємо таку матрицю:
\begin{equation}
C_1' = \begin{array}{c|ccccc|c|c|}
  & 0 & 1 & 3 & 4 & 5 & h_i & \al_r\\
\hline
1 & 0 & x & 0 & 3 & 0 & 0 & 3 \\
2 & x & 2 & 2 & 0 & 3 & 0 & 2\\
3 & 3 & 3 & x & 0 & 2 & 0 & 2\\
4 & 4 & 1 & 0 & x & 0 & 0 & 0\\
5 & 8 & 0 & 1 & 0 & x & 0 & 0\\
\hline
H_j & 0 & 0 & 0 & 0 & 0  \\
\cline{1-6}
\beta_m & 3 & 1 & 1 & 0 & 2 
\end{array}
\end{equation}
Знайдемо оцінку $G_1\cb{\xi}$ за формулою:
\begin{equation}
	G_1^2\cb{\xi} = G_0\cb{\xi} + \suml_{i} h_i + \suml_{j} H_j = 20
\end{equation}
Розіб’ємо підмножину за таким самим алгоритмом. $\Theta_{1,0} = 6$:
\begin{center}
\begin{tikzpicture}
\node[circle,draw] (c1) {$G_0$};
\node[circle,draw,below left=of c1] (c11) {$G_1'$};
\node[circle,draw,below right=of c1] (c12) {$G_2'$};
\node[circle,draw,below left=of c11] (c111) {$G_1^2$};
\node[circle,draw,below right=of c11] (c112) {$G_2^2$};
\path[-]
(c1) edge node[above,rotate=45]{(0,2)}(c11)
(c1) edge node[above,rotate=-45]{$\overline{(0,2)}$}(c12)
(c11) edge node[above,rotate=45]{(1,0)}(c111)
(c11) edge node[above,rotate=-45]{$\overline{(1,0)}$}(c112)
;
\end{tikzpicture}
\end{center}
Отримали оцінку : $G_2^2\cb{\xi} = G_1\cb{\xi} + \Theta_{1,0} = 26$.\\
Знову викреслюємо стовпчик і рядок за переходом (1,0). Також забороняємо перехід (2,1), оскільки він є транзитивним для $(2,0)\to(0,1)$.
\begin{equation}
C_1^2 = \begin{array}{c|cccc|c|c|}
&1 & 3 & 4 & 5 & h_i & \al_r\\
\hline
2 & x & 2 & 0 & 3 & 0 & 2\\
3 & 4 & x & 0 & 2 & 0 & 2\\
4 & 1 & 0 & x & 0 & 0 & 0\\
5 & 0 & 1 & 0 & x & 0 & 0\\
\hline
H_j & 0 & 0 & 0 & 0\\
\cline{1-5}
\beta_m & 1 & 1 & 0 & 2
\end{array}
\end{equation}
Отже, $G_1^2\cb{\xi}=G_1\cb{\xi} + \suml_i h_j + \suml_j H_j = 20$.\\
Знайдемо нове розбиття. $\Theta_{2,4} = \Theta_{3,4} = \Theta_{4,5} = 2$. Немає різниці, яку обирати, тому оберемо (2,4).\\
\begin{center}
\begin{tikzpicture}
\node[circle,draw] (c1) {$G_0$};
\node[circle,draw,below left=of c1] (c11) {$G_1'$};
\node[circle,draw,below right=of c1] (c12) {$G_2'$};
\node[circle,draw,below left=of c11] (c111) {$G_1^2$};
\node[circle,draw,below right=of c11] (c112) {$G_2^2$};
\node[circle,draw,below left=of c111] (c1111) {$G_1^3$};
\node[circle,draw,below right=of c111] (c1112) {$G_2^3$};
\path[-]
(c1) edge node[above,rotate=45]{(0,2)}(c11)
(c1) edge node[above,rotate=-45]{$\overline{(0,2)}$}(c12)
(c11) edge node[above,rotate=45]{(1,0)}(c111)
(c11) edge node[above,rotate=-45]{$\overline{(1,0)}$}(c112)
(c111) edge node[above,rotate=45]{(2,4)}(c1111)
(c111) edge node[above,rotate=-45]{$\overline{(2,4)}$}(c1112)
;
\end{tikzpicture}
\end{center}
Отримали оцінку $G_2^3\cb{\xi} = G_1^2\cb{\xi} + \Theta_{2,4} = 22$\\
Знову викреслюємо рядок і стовпчик за переходом (2,4) та забороняємо (4,1) як транзитивний від $(4,2)\to(2,1)$.
\begin{equation}
C_1^3=\begin{array}{c|ccc|c}
 & 1 & 3 & 5 & h_i\\
\hline
3 & 4 & x & 2 & 2\\
4 & x & 0 & 0 & 0\\
5 & 0 & 1 & x & 0 
\end{array}
\end{equation}
Віднімаємо двійку та отримуємо:
\begin{equation}
C_1^3=\begin{array}{c|ccc|c|c|}
 & 1 & 3 & 5 & h_i & \al_i\\
\hline
3 & 2 & x & 0 & 2 & 2\\
4 & x & 0 & 0 & 0 & 0\\
5 & 0 & 1 & x & 0 & 1\\
\hline
H_J & 0 & 0 & 0 \\
\cline{1-4}
\beta_j & 2 & 1 & 0
\end{array}
\end{equation}
Отже, $G_1^3\cb{\xi} = G_1^2\cb{\xi} + \suml_{i} h_i + \suml_{j} H_j = 20 + 2 = 22$\\
Оскільки оцінки у обох варіантах однакові, оберемо $G_2^3$ та розберемося з тим, що відбувається, коли ми заперечуємо перехід.\\
Розглянемо гілку $G_2^3$ та побудуємо для неї матрицю. Для цього просто забороняємо перехід $(2,4)$.:
\begin{equation}
C^3_2 = \begin{array}{c|cccc|c}
  & 1 & 3 & 4 & 5 & h_i\\
\hline
2 & x & 2 & x & 3 & 2\\
3 & 4 & x & 0 & 2 & 0\\
4 & 1 & 0 & x & 0 & 0\\
5 & 0 & 1 & 0 & x & 0\\
\hline
\end{array}
\end{equation}
Після віднімання:
\begin{equation}
C^3_2 = \begin{array}{c|cccc|c|c}
  & 1 & 3 & 4 & 5 & h_i & \al_r\\
\hline
2 & x & 0 & x & 1 & 2 & 1\\
3 & 4 & x & 0 & 2 & 0 & 2\\
4 & 1 & 0 & x & 0 & 0 & 0\\
5 & 0 & 1 & 0 & x & 0 & 0 \\
\hline
H_j & 0 & 0 & 0 & 0\\
\cline{1-5}
\beta_m & 1 & 0 & 0 & 1
\end{array}
\end{equation}
Знаходимо перехід. $\Theta_{3,4}=\Theta_{5,4} = 2$
\begin{center}
\begin{tikzpicture}
\node[circle,draw] (c1) {$G_0$};
\node[circle,draw,below left=of c1] (c11) {$G_1'$};
\node[circle,draw,below right=of c1] (c12) {$G_2'$};
\node[circle,draw,below left=of c11] (c111) {$G_1^2$};
\node[circle,draw,below right=of c11] (c112) {$G_2^2$};
\node[circle,draw,below left=of c111] (c1111) {$G_1^3$};
\node[circle,draw,below right=of c111] (c1112) {$G_2^3$};
\node[circle,draw,below left=of c1112] (c11121) {$G_1^4$};
\node[circle,draw,below right=of c1112] (c11122) {$G_2^4$};
\path[-]
(c1) edge node[above,rotate=45]{(0,2)}(c11)
(c1) edge node[above,rotate=-45]{$\overline{(0,2)}$}(c12)
(c11) edge node[above,rotate=45]{(1,0)}(c111)
(c11) edge node[above,rotate=-45]{$\overline{(1,0)}$}(c112)
(c111) edge node[above,rotate=45]{(2,4)}(c1111)
(c111) edge node[above,rotate=-45]{$\overline{(2,4)}$}(c1112)
(c1112) edge node[above,rotate=45]{(3,4)}(c11121)
(c1112) edge node[above,rotate=-45]{$\overline{(3,4)}$}(c11122)
;
\end{tikzpicture}
\end{center}
Розглянемо перехід $G_1^4$ та побудуємо для нього матрицю:
\begin{equation}
C_1^4 =  \begin{array} {c|ccc|c|c}
  & 1 & 3 & 5 & h_i & \al_r\\
\hline
2 & x & 0 & 1 & 0 & 1\\
4 & 1 & x & 0 & 0 & 1\\
5 & 0 & 1 & x & 0 & 1\\
\hline
H_j & 0 & 0 & 0 \\
\cline{1-4}
\beta_m & 1 & 1 & 1
\end{array}
\end{equation}
Оцінка вийшла $G_1^4\cb{\xi} = G_2^3\cb{\xi} + \suml_i h_i + \suml_j H_j =  22$\\
Знайдемо перехід. $\Theta_{2,3}=\Theta_{4,5} = \Theta_{5,1} = 2$
\begin{center}
\begin{tikzpicture}
\node[circle,draw] (c1) {$G_0$};
\node[circle,draw,below left=of c1] (c11) {$G_1'$};
\node[circle,draw,below right=of c1] (c12) {$G_2'$};
\node[circle,draw,below left=of c11] (c111) {$G_1^2$};
\node[circle,draw,below right=of c11] (c112) {$G_2^2$};
\node[circle,draw,below left=of c111] (c1111) {$G_1^3$};
\node[circle,draw,below right=of c111] (c1112) {$G_2^3$};
\node[circle,draw,below left=of c1112] (c11121) {$G_1^4$};
\node[circle,draw,below right=of c1112] (c11122) {$G_2^4$};
\node[circle,draw,below left=of c11121] (c111211) {$G_1^5$};
\node[circle,draw,below right=of c11121] (c111212) {$G_2^5$};
\path[-]
(c1) edge node[above,rotate=45]{(0,2)}(c11)
(c1) edge node[above,rotate=-45]{$\overline{(0,2)}$}(c12)
(c11) edge node[above,rotate=45]{(1,0)}(c111)
(c11) edge node[above,rotate=-45]{$\overline{(1,0)}$}(c112)
(c111) edge node[above,rotate=45]{(2,4)}(c1111)
(c111) edge node[above,rotate=-45]{$\overline{(2,4)}$}(c1112)
(c1112) edge node[above,rotate=45]{(3,4)}(c11121)
(c1112) edge node[above,rotate=-45]{$\overline{(3,4)}$}(c11122)
(c11121) edge node[above,rotate=45]{(2,3)}(c111211)
(c11121) edge node[above,rotate=-45]{$\overline{(2,3)}$}(c111212)
;
\end{tikzpicture}
\end{center}
Отримали оцінку $G_2^5\cb{\xi}=G_1^4 + \Theta_{2,3} = 24$.\\
Розглянемо матрицю для $G_1^5$
\begin{equation}
C_1^5=\begin{array}{c|cc|c}
  & 1 & 5 & h_i\\
\hline
4 & x & 0 & 0\\
5 & 0 & x & 0 \\
\hline
H_j & 0 & 0
\end{array}
\end{equation}
Отримали оцінку $G_1^5\cb{\xi}=G_1^4\cb{\xi} + \suml_i h_i + \suml_{j} H_j = 22$. Отже, ми знайшли оптимальний шлях, оскільки далі є лише ті переходи, які потрібні для завершення шляху:
\begin{center}
\begin{tikzpicture}
\node[circle,draw] (c1) {$G_0$};
\node[circle,draw,below left=of c1] (c11) {$G_1'$};
\node[circle,draw,below right=of c1] (c12) {$G_2'$};
\node[circle,draw,below left=of c11] (c111) {$G_1^2$};
\node[circle,draw,below right=of c11] (c112) {$G_2^2$};
\node[circle,draw,below left=of c111] (c1111) {$G_1^3$};
\node[circle,draw,below right=of c111] (c1112) {$G_2^3$};
\node[circle,draw,below left=of c1112] (c11121) {$G_1^4$};
\node[circle,draw,below right=of c1112] (c11122) {$G_2^4$};
\node[circle,draw,below left=of c11121] (c111211) {$G_1^5$};
\node[circle,draw,below right=of c11121] (c111212) {$G_2^5$};
\node[circle,draw,left=of c111211] (c5) {$G_1^6$};
\node[circle,draw,left=of c5] (c6) {$G_1^7$};
\path[-]
(c1) edge node[above,rotate=45]{(0,2)}(c11)
(c1) edge node[above,rotate=-45]{$\overline{(0,2)}$}(c12)
(c11) edge node[above,rotate=45]{(1,0)}(c111)
(c11) edge node[above,rotate=-45]{$\overline{(1,0)}$}(c112)
(c111) edge node[above,rotate=45]{(2,4)}(c1111)
(c111) edge node[above,rotate=-45]{$\overline{(2,4)}$}(c1112)
(c1112) edge node[above,rotate=45]{(3,4)}(c11121)
(c1112) edge node[above,rotate=-45]{$\overline{(3,4)}$}(c11122)
(c11121) edge node[above,rotate=45]{(2,3)}(c111211)
(c11121) edge node[above,rotate=-45]{$\overline{(2,3)}$}(c111212)
(c111211) edge node[above]{(4,5)} (c5)
(c5) edge node[above]{(5,1)} (c6)
;
\end{tikzpicture}
\end{center}
Намалюємо графік руху:
\begin{center}
\begin{tikzpicture}
\node (c0) {0};
\node[right=of c0] (c2) {2};
\node[below=of c2] (c3) {3};
\node[below left=of c3] (c4) {4};
\node[above left=of c4] (c5) {5};
\node[above=of c5] (c1) {1};
\path[->]
(c0) edge node[above]{1} (c2)
(c2) edge node[right]{6} (c3)
(c3) edge node[below right]{3} (c4)
(c4) edge node[below left]{6} (c5)
(c5) edge node[left]{4} (c1)
(c1) edge node[above]{2} (c0)
;
\end{tikzpicture}
\end{center}
\end{exs}
\section{Метод послідовного аналізу та відсів варіантів}
Ідея методу: розв’язується задача дискретної оптимізації і на етапі побудови дерева варіантів відкидаються недопустимі варіанти або неефективні з точки зору цільової функції. Для цього будуються дві процедури відсіву $\omg_1$ - відсів за обмеженнями, а \nomg2 - відсів по значенням цільової функції. 
\begin{eqnarray}
&\min f\\
&g_p(x) \leq g_p^\ast, &\pfoq\\
&g_p(x) \geq g_p^\ast, &\pfrm{q+1}Q\\
&x\in \mX
\end{eqnarray}
Нехай $x^p$ забезпечує оптимум при фіксованому $x_j=x_{j_{k_j}}$. 
\subsection{Процедура \nomg1}
\begin{eqnarray}
&g_p\cb{x^p|x_j=x_{j_{k_j}}} > g_p^\ast,&\pfoq\label{tr:6:1}\\
&g_p\cb{x^p|x_j=x_{j_{k_j}}}  \geq g_p^\ast,& \pfrm{q+1}Q\label{tr:6:2}
\end{eqnarray}
Використовуючи процедури відсіву \eqref{tr:6:1} та \eqref{tr:6:2} виконуємо процедуру \nomg1 для всіх обмежень та для всіх змінних. Використовуємо процедуру \nomg1 до того часу, поки відсів не закічиться. Множину варіантів, що залишилася позначимо як $\mX(l)$.
\begin{tver}
Якщо при двох послідовних використань процедури \nomg1 множина точок оптимуму не змінюється, використання процедури \nomg1 закінчується
\end{tver}
\begin{tver}
Допустима множина при використанні процедури \nomg1 не зменшується.
\end{tver}
При використанні процедури \nomg1 до початкової множини можливі наступні випадки:
\begin{itemize}
\item $\mX(l)=\emptyset$ : задача нерозв’язна.
\item $\mX(l)\neq\emptyset$ в цьому випадку йдемо на процедуру \nomg2
\end{itemize}
\subsection{Процедура \nomg2}
Будуємо допоміжну задачу.
\begin{eqnarray}
\min f(x)\\
f(x) \leq f_1^\ast\\
x\in\mX(l)
\end{eqnarray}
Де, $f_1^\ast$ - це порог відсіву, який визначається так:
\begin{equation}
f_1^* = \cfrac12\cb{\min\limits_{x\in\mX(l)} f(x) + \max\limits_{x\in\mX(l)} f(x)}
\end{equation}
Нехай $x^p$ забезпечує оптимум функції $f(x)$ при фіксованому $x_j=x_{j_{k_j}}$. Записуємо умову оцінки
\begin{equation}\label{tr:6:3}
f\cb{x^p|x_j=x_{j_{k_j}}} > f_1^\ast
\end{equation}
Тобто $x_{j_{k_j}}$ відсівається, якщо виконується умова \eqref{tr:6:3}. Виконуємо процедуру \nomg2 для всіх змінних. Якщо відсів стався, переходимо на процедуру \nomg1. В іншому випадку необхідно ввести новий поріг:
\begin{equation}
f_2^\ast = \cfrac12\cb{f_1^* + \min\limits_{x\in\mX(l)} f(x)}
\end{equation}
Виконуємо процедуру \nomg2 з новим порогом і, якщо відсів відбувся, йдемо на процедуру \nomg1. Якщо відсіву знову не відбулося, то збільшуємо поріг.\\
Якщо після використання процедури \nomg1 множина варіантів, що залишилася, порожня, необхідно перейти на процедуру \nomg2 і розширити множину варіантів. Для цього вводимо новий поріг:
\begin{equation}
f_3^\ast = \cfrac12\cb{f_2^\ast + \max\limits_{x\in\mX(l)} f(x)}
\end{equation}
Використовуємо процедури \nomg1 та \nomg2 багатократно до того часу, поки не буде знайдемо оптимальний розв’язок. При чому, на k-тому кроку відсів відбувається за умовою:
\begin{eqnarray}
f\cb{x^p|x_j=x_{j_{k_j}}} > f_k^\ast
\end{eqnarray}
Де $f_k^\ast$ - поріг відсіву
\begin{equation}
f_k^\ast = \system{\cfrac12\cb{f_{k-1}^\ast +\max\limits_{x\in\mX(l)} f(x)},& \mX(l)\uparrow \\
\cfrac12\cb{f_{k-1}^\ast +\min\limits_{x\in\mX(l)} f(x)},& \mX(l)\downarrow\\
\cfrac12\cb{f_{k-1}^\ast + f_{k-2}^\ast},&\mX(l)\uparrow after\mX(l)\downarrow or~reverse
}
\end{equation}
\section{Метод послідовного відсіву варіантів цілого програмування}
Запишемо модель:
\begin{eqnarray}
&\max \sumjon \cj\xj &\\
&\sumjon \aij \xj \leq \bi,&\ifom\\
&\xj\geq 0,&\jfon\label{tr:6:4}\\
&x_j\in\mZ\label{tr:6:5}&
\end{eqnarray}
Для кожної змінної визначаємо найменше і найнижче значення. Умови \eqref{tr:6:4} та \eqref{tr:6:5} у загальному виді:
\begin{equation}
0 \leq d_j^{(1)} \leq \xj \leq d_j^{(2)}
\end{equation}
$d_j^{(1)}=0$ - нижня границя, а верхня границя визначається як:
\begin{equation}
d_j^{(2)} = \min\set{\bb{\cfrac{b_i}{\aij}}_{\aij>0}},a_{ik}\geq0
\end{equation}