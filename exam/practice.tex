Список задач з посиланнями на розв’язки:
\begin{enumerate}
\item \sout{Багатокритеріальні задачі}
\item Задача комівояжера (Приклад \ref{exam:pr:1})
\item Задача на метод гілок та меж (Приклад \ref{exs:pr:2}).
\item Задачі на метод послідовного аналізу та відсіву варіантів (Приклад \ref{exs:pr:3})
\item Задачі на метод Куна-Такера
\item Задачі на метод можливих напрямів з лінійними обмеженнями
\item Задачі на метод можливих напрямів з нелінійними обмеженнями
\item Задача розподілу ресурсів
\item Задача керування ресурсами
\item \sout{Геометричні задачі}
\end{enumerate}
\begin{exs}\label{exs:pr:2}
Роз’язати методом гілок та меж таку задачу:
\begin{eqnarray}
	&\max x_1+x_2\\
	&6x_1+5x_2 \leq 20\\
	&2x_1+3x_2\leq 10 \\
	&x_1,x_2\geq 0\\
	&x_1,x_2\in\mZ
\end{eqnarray}
Приведемо до канонічного виду:
\begin{eqnarray}
	&\max x_1 + x_2 \\
	&6x_1 + 5x_2 + x_3 = 20\\
	&2x_1 + 3x_2 + x_4 = 10\\
	&x_1,x_2\geq 0
\end{eqnarray}
Спочатку знаходимо оптимальний розв’язок задачі лінійного програмування за допомогою симплекс методу. 
\begin{equation}
	\begin{array}{|c|c|c|c|c|c|c|}
		\hline
		  &     &     & 1   & 1   &     & \\
		\hline
		c & B_x & A_0 & A_1 & A_2 & A_3 & A_4 \\
		\hline
		1 & x_1 & \cfrac54 & 1 & 0 & \cfrac38 & \cfrac58\\
		\hline
		1 & x_2 & \cfrac52 & 0 & 1 & -\cfrac14 & \cfrac34\\
		\hline
		& \Delta & \cfrac{15}4& 0 & 0 & \cfrac18 & \cfrac{11}8\\
		\hline
	\end{array}
\end{equation}
Обчислюємо оцінку на множині
\begin{equation}
	G_0:\quad \xi\cb{G_0} = \bb{f(x^0)} = \bb{\cfrac{15}4} = 3
\end{equation}
Оскільки $x_1$ не цілий, то розбиваємо по цій компоненті 
\begin{equation}
	G_0:\quad G_0 = G^1_1\cup G_2^1,\quad G_1^1 = G_0 \cap \set{x_1\leq 1}
\end{equation}
Запишемо обмеження $x_1\leq 1$ у вигляді $x_1+x_5=1$\\
Дописуємо рядок $x_5$ і стовпець $A_5=(0,0,1)$ до симплекс таблиці. Віднімаємо від $x_5$ рядок $x_1$ і отримуємо рядок $x_5'$ в таблиці. Далі розв’язуємо задачу двоїстим симплекс методом:
\begin{equation}
	\begin{array}{|c|c|c|c|c|c|c|c|}
		\hline
		  &     &     & 1   & 1   &     &  &\\
		\hline
		c & B_x & A_0 & A_1 & A_2 & A_3 & A_4 & A_5\\
		\hline
		1 & x_1 & \cfrac54 & 1 & 0 & \cfrac38 & \cfrac58 & 0\\
		\hline
		1 & x_2 & \cfrac52 & 0 & 1 & -\cfrac14 & \cfrac34 & 0\\
		\hline
		  & x_5 & 1 & 1 & 0 & 0 & 0 & 1\\
		\hline
		  & x_5'& -\cfrac14& 0 & 0 & -\cfrac38 & -\cfrac58& 1\\
		\hline
	\end{array}
\end{equation}
Виконавши ітерацію симплекс методом отримуємо новий розв’язок $x_1^1 = \bb{1;\cfrac83}$. Цей розв’язок також не цілочисельний, його оцінка $\xi\cb{G_1^1} = 3$. Тепер знаходимо розв’язок на множині $G_1^2 = G_0 \cap \cb{x_1\geq 2}$. Це виконується аналогічним чином до попереднього кроку з $x_5$. Отримали нероз’язну задачу. Отже, $\xi\cb{G_2^1} = -\infty$.\\
Інші ітерації виконуються так само.
\end{exs}