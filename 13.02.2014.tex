\section{Задачі багатокритерійної оптимізації}\marginpar{\framebox{13.02.2014}}
Математична модель:
\begin{eqnarray}
&\max f_i(x), i\in I_1=\set{1,\ldots,m}\label{tr:1:1}\\
&\min f_i(x), i\in I_2=\set{m+1,\ldots,M}\label{tr:1:2}
\end{eqnarray}
Введемо наступні відношення переваги та еквівалентності на множині альтернатив.\\
Відношення {\bf нестрогого переваги} $x\succeq y$ виконується, якщо:
\begin{equation}\label{tr:1:3}
\system{
f_i(x)\geq f_i(y),i\in I_1\\
f_i(x)\leq f_i(y),i\in I_2
}
\end{equation}

Відношення {\bf строгої переваги} $x\succ y$, якщо виконуються попередні обмеження \eqref{tr:1:3}
і одне виконується строго.\\
Відношення {\bf еквівалентності} $x\sim y$, якщо $f_i(x)=f_i(y),i\in I=\set{1,\ldots,M}$\\
Альтернатива $x_0$ називається {\bf паретто оптимальною} або {\bf ефективною}, якщо не існує такої альтернативи $x'$, для якої виконується умова:
\begin{equation}
\system{f_i(x')\geq f_i(x_0),i\in I_1\\f_i(x')\leq f_i(x_0),i\in I_2}
\end{equation}
\begin{teor}
Дві ефективні альтернативи $x_1$ та $x_2$ або є непорівнянними, або еквівалентні.
\end{teor}
\begin{proof}
Нехай ці дві альтернативи є порівняними між собою. Так, як $x_1$ ефективна альтернатива, то для неї виконується умова:
\begin{equation}\label{tr:1:4}
\system{f_i(x_1)\geq f_i(x_2),i\in I_1\\f_i(x_1)\leq f_i(x_2),i\in I_2}
\end{equation}
Так як $x_2$ ефективна альтернатива, то для неї виконується умова:
\begin{equation}\label{tr:1:5}
\system{f_i(x_2)\geq f_i(x_1),i\in I_1\\f_i(x_2)\leq f_i(x_1),i\in I_2}
\end{equation}
Порівнюючи умови \eqref{tr:1:4} та \eqref{tr:1:5} отримуємо, що:
\begin{equation}
f_i(x_1)=f_i(x_2),i\in I=\set{1,\ldots,M}
\end{equation}
\end{proof}
Нехай задані два вектора критерії $f^1(x)$ та $f^2(x)$ називаються {\bf еквівалентними} ($f^1(x)\sim f^2(x)$), якщо вони породжують однакові відношення еквівалентності та предпочтения.\\
Позначимо через $f^0_i = \max f_i(x),i\in I_1$, через $f^0_i = \min f_i(x),i\in I_2$\\
Розглянемо довільну альтернативу $x^*$ і ступінь близькості до оптимальних значень оцінимо наступним чином:
\begin{equation}
\Delta f_i(x^*)=\system{f_i^0-f_i(x),i\in I_1\\ f_i(x)-f_i^0,i\in I_2}
\end{equation}
Найкращою компромісною альтернативою варто рахувати таку альтернативу, для якої ступінь відхилення від оптимальних значень по всім критеріям мінімальна.\\
Так як цільові функції мають різну фізичну розмірність, то зручно розглядати не саму множину цільових функцій, а еквівалентну їй множину $W(x)=\set{w_i(x)|i\in I}$, де $w_i(x)$ - це монотонні перетворення, що приводять цільові функції до безрозмірного виду. Це монотонне перетворення має задовольняти наступним вимогам:
\begin{itemize}
\item Мати єдину точку відліку та однакову шкалу зміни значень;
\item Зберігається множина ефективних альтернатив;
\item Враховувати необхідність мінімізації відхилень від оптимальних значень по всім критеріям.
\end{itemize}
Монотонне перетворення визначається наступним чином:
\begin{eqnarray}
&w_i(x) = \system{\cfrac{f_i^0 - f_i(x)}{f_i - f_{i,min}},i\in I_1 \\ \cfrac{f_i(x)-f_i^0}{f_{i,max}-f_i^0},i\in I_2}\label{tr:1:6} \\
&0<w_i(x)<1 \nonumber
\end{eqnarray}
Ще одна формула для монотонного перетворення:
\begin{equation}
w_i(x) = \system{
\cfrac{f_i^0-f_i(x)}{f_i^0},i\in I_1 \\
\cfrac{f_i(x)-f_i^0}{f_i^0},i\in I_2
}
\end{equation}
В даному випадку шкала не одинична.
\begin{teor}
Для будь-якої альтернативи існують вагові коефіцієнти $\exists \rho_i\geq 0,\suml_{i=1}^M \rho_i=1$ і параметри $k_0$ такий, що 
\begin{equation}
\rho_i\cdot w_i(x) = k_0
\end{equation}
\end{teor}
\begin{teor}
Якщо для двох альтернатив $x'$ та $x''$ вагові коефіцієнти співпадають, то виконується умова:
\begin{equation}
\system{
&k_0'=\gamma k_0'' \\
&w_i(x')=\gamma w_i(x'')
}
\end{equation}
Де $\gamma$ коефіцієнт пропорційності.
\end{teor}
\begin{teor}
Для того, щоб альтернатива $x_0$ була ефективною, досить того, щоб вона була єдиними розв’язком системи:
\begin{equation}
\rho_i w_i(x) \leq k_{0,min}
\end{equation}
\end{teor}
\begin{teor}\label{tr:1:7}
Якщо $x_0$ ефективна альтернатива, то для неї відносно взважені відхилення від оптимальних значень однакові та мінімальні.
\end{teor}
\subsection{Метод обмежень}
Від багатокритеріальної задачі \eqref{tr:1:1}-\eqref{tr:1:2} переходимо до однокритеріальної задачі
\begin{eqnarray}
&\min k_0 \\
&\rho_i w_i(x)\leq k_0,i\in I \label{tr:1:8}
\end{eqnarray}
Використовуючи умову для монотонного перетворення \eqref{tr:1:6} перетворюємо систему \eqref{tr:1:8} у:
\begin{eqnarray}
f_i(x) \geq f_i^0 - \cfrac{k_0}{\rho_i}\cb{f_i^0 - f_{i,min}},i\in I_1\label{tr:1:9}\\
f_i(x) \leq f_i^0 - \cfrac{k_0}{\rho_i}\cb{f_{i,max}-f_i^0},i\in I_2\label{tr:1:10}
\end{eqnarray}
Необхідно знайти мініальний $k_0$ при якому система \eqref{tr:1:9}-\eqref{tr:1:10} буде сумісною.
\begin{description}
\item[Крок 1.] $k_0=k_0(0)$
\item[Крок 2.] Зменшуємо $k_0(1)=k_0(0)-\Delta k_0$. Перевіряємо, чи сумісні системи. Якщо сумісні, то на наступний крок;\\
$\vdots$
\item[Крок n.] $k_0(n-1)=k_(n-2)-\Delta k_0$
\end{description}
Нехай на n-тій ітерації система сумісна, а на n+1-шій вже не сумісна. Тоді нашим розв’язком буде $k_0(n-1)$. \\
Так як загальний метод не залежить від функціональної залежності, то для кожного виду залежності потрібно знаходити свій спосіб перевірки сумісності систем \eqref{tr:1:9}-\eqref{tr:1:10}
\begin{exs}
Записуємо математичну модель:
\begin{eqnarray*}
&f_1(x) = 3x_1+2x_2\to\max \\
&f_2(x) = x_1-3x_2\to\min \\
&-2x_1+x_2 \leq 4 \\
&x_1 + 2x_2 \geq 8 \\
&x_1 + 2x_2 \leq 20 \\
&-x_1 + 4x_2 \geq 12 \\
&x_1 \leq 6 \\
&x_1,x_2 \geq 0 \\
&\rho_1=\rho_2=\frac12
\end{eqnarray*}
Знайдемо потрібні розв’язки для перетворення \eqref{tr:1:6} графічним методом.\\
%а тут малюнок
\begin{tikzpicture}[scale=0.5]
% \xyplot
% \markxyplot
\draw 
(0,4)node[circle,fill,scale=0.3] (A){}
(2.4,8.8)node[circle,fill,scale=0.3] (B){}
(6,7)node[circle,fill,scale=0.3] (C){}
(6,4.5)node[circle,fill,label=below right:$D$,scale=0.3] (D){}
(1.333,3.333)node[circle,fill,scale=0.3] (E){}; 
\draw ($ (B)!2!(A) $) -- ($ (A)!1.5!(B) $) node[black,above right]{(1)};
\draw ($ (E)!2!(A) $) -- ($ (A)!7!(E) $) node[black,below right]{(2)};
\draw ($ (E)!3!(D) $) -- ($ (D)!2!(E) $) node[black,below right]{(4)};
\draw ($ (B)!4!(C) $) -- ($ (C)!2!(B) $)  node[black,above left]{(3)};
\draw ($ (C)!3.5!(D) $) -- ($ (D)!2!(C) $)  node[black,above]{(5)};
\draw[red] (A) -- (B) -- (C) -- (D) -- (E) -- (A);
\node[above left=0.3cm of A] {$A$};
\node[above=0.1cm of B.140] {$B$};
\node[above right=0cm of C] {$C$};
\node[below=0.3cm of E] {$E$};
\end{tikzpicture} \\
Отримали випуклий багатокутник з вершинами:\\
$A(0,4)$;$B\cb{2.4,\cfrac{44}5}$;$C(6,7)$;$D(6,4.5)$;$E\cb{\cfrac43,\cfrac{10}3}$\\
Будуємо вектори нормальні.\\
$f_1^0(C) = 32;f_{1,min}(A)=8$\\
$f_2^0(B)=-24;f_{2,max}(D)=-7.5$\\
Отже, тепер можливо використати монотонне перетворення \eqref{tr:1:6}.\\
\begin{eqnarray*}
&w_1(x)=\cfrac{32-(3x_1+2x_2)}{32-8}\\
&w_2(x)=\cfrac{x_1-3x_2+24}{-7.5+24}
\end{eqnarray*}
Тепер ми можемо перейти до однокритеріальної задачі:
\begin{eqnarray*}
&\min k_0=x_3 \\
&\cfrac12 \cdot \cfrac{32-3x_1-2x_2}{32-8} \leq x_3 \\
&\cfrac12 \cdot \cfrac{x_1-3x_2+24}{-7.5+24} \leq x_3 \\
&-2x_1+x_2 \leq 4 \\
&x_1 + 2x_2 \geq 8 \\
&x_1 + 2x_2 \leq 20 \\
&-x_1 + 4x_2 \geq 12 \\
&x_1 \leq 6 \\
&x_1,x_2 \geq 0 \\
&\rho_1=\rho_2=\frac12
\end{eqnarray*}
Зрозуміло, що кращій компромісний розв’язок буде знаходитися на прямій $BC$. Тобто $x_1+2x_2=20$.\\
А з теореми \ref{tr:1:7} можна отримати $\cfrac{32-3x_1-2x_2}{32-8}=\cfrac{x_1-3x_2+24}{-7.5+24}$\\
Розв’язавши дану систему можна знайти $x_1,x_2$.\\
$x_1^0=3.8,x_2^0=8.1$
\end{exs}