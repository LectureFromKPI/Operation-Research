\section{Знаходження найкращої альтернативи в задачах багатокритерійного вибору}\marginpar{\framebox{27.02.2014}}
Постановка задачі:
Задані критерії та відношення переваги та вагові коефіцієнти важливості:
\begin{eqnarray}
&f_i,\jfon\\
&R_j,\jfon\\
&\omg_j\geq 0,\jfon
\end{eqnarray}
Необхідно знайти найкращу альтернативу. Для розв’язку цієї задачі необхідно визначити ефективний спосіб згортання векторного критерію в скалярний.
Способи:
\begin{itemize}
\item $Q_1=R_1\cap R_2\ldots\cap R_n$
\item $Q_2=\sumjon \omg_j R_j$
\end{itemize}
Отже, необхідно знайти найкращу альтернативу за обома згортками. \\
\begin{description}
\item[Крок 1.] Будуємо функцію приналежності заданих відношень переваги.
\begin{equation}
\mu_{R_j} (x,y) = \system{1&,x \succ y,x\sim y\\0&}
\end{equation}
\item[Крок 2.] Визначаємо першу згортку та будуємо функцію приналежності.
\begin{eqnarray}
&Q_1=\bcapl_{j=1}^n R_j\\
&\mu_{Q_1}(x,y) = \min\set{\mu_{R_1}(x,y),\ldots,\mu_{R_n}(x,y)}
\end{eqnarray}
\item[Крок 3.]  Визначаємо відношення строгої переваги, тобто ступінь, з якою альтернатива $x$ краще за альтернативу $y$ і будуємо функцію приналежності $Q_1^S$.
\begin{equation}
\mu_{Q_1^S}(x,y) = \max\set{0;\mu_{Q_1}(x,y)-\mu_{Q_1}(y,x)}
\end{equation}
\item[Крок 4.]  Визначаємо множину альтернатив, що не домінують, тобто $Q_1^n$. Тобто, ступінь з якої альтернатива $x$ не домінується ніякої іншою альтернативою.
\begin{equation}
\mu_{Q_1^n} = 1 - \max\set{\mu_{Q_1^S}(y,x)}
\end{equation}
\item[Крок 5.]  Визначаємо другу згортку за формулою $Q_2=\sumjon \omg_j R_j$ і будуємо функцію приналежності: 
\begin{equation}
\mu_{Q_2}(x,y) = \sumjon \omg_j \mu_{R_j}(x,y)
\end{equation}
\item[Крок 6.]  Визначаємо відношення строгої переваги за другою згорткою $Q_2^S$ і будуємо функцію приналежності.
\begin{equation}
\mu_{Q_2^S}(x,y) = \max\set{0;\mu_{Q_2}(x,y)-\mu_{Q_2}(y,x)}
\end{equation}
\item[Крок 7.]  Визначаємо множину альтернатив, що не домінують за другою згорткою $Q_2^n$ і будуємо функцію приналежності.
\begin{equation}
\mu_{Q_2^n} = 1 - \max\set{\mu_{Q_2^S}(y,x)}
\end{equation}
\item[Крок 8.]  Визначаємо множину альтернатив, що не домінують за обома згортками $Q^n(X) = Q_1^n(X) \cap Q_2^n(X)$ і будуємо функцію приналежності:
\begin{equation}
\mu_{Q^n}(x) = \min\set{\mu_{Q_1^n}(x),\mu_{Q_2^n}(x)}
\end{equation}
\item[Крок 9.]  Визначаємо найкращу альтернативу за обома згортками. Найкращою альтернативою є така альтернатива $x_0$, для якої виконується:
\begin{equation}
\mu_{Q^n}(x_0) = \max\set{\mu_{Q^n}(x)}
\end{equation}
\end{description}
Якщо ступінь альтернативи, що не домінує є 1, то це \textbf{чітко не домінуєма альтернатива}.
\begin{exs}
Розглядаються наступні альтернативи при купівлі квартири:
\begin{itemize}
\item $x_1$ - купити квартиру в центрі Києва;
\item $x_2$ - купити квартиру на околоці Києва;
\item $x_3$ - купити квартиру десь взагалі на краю Києва.
\end{itemize}
Вибір буде відбуватися за наступними критеріями:
\begin{itemize}
\item Властивість квартири;
\item Затрати на дорогу;
\item Загазованість навколишнього середовища;
\item Досуг після роботи.
\end{itemize}
Задані коефіцієнти вагомості $\omg_1=0.4,\omg_2=0.25,\omg_3 = 0.2,\omg_4 = 0.15$.\\
\begin{eqnarray*}
R_1:& x_2\succ x_1, x_3 \succ x_2\\
R_2:& x_2 \succ x_3, x_2\succ x_1, x_1 \sim x_3\\
R_3:& x_2\sim x_3, x_2 \succ x_1\\
R_4:& x_1 \succ x_2, x_2 \sim x_3
\end{eqnarray*}
Будуємо функцію приналежності:\\
$\mu_{R_1}(x_i,x_j)$=\begin{tabular}{c|c|c|c}
 & $x_1$ & $x_2$ & $x_3$ \\ 
\hline 
$x_1$ & 1 & 0 & 0 \\ 
\hline 
$x_2$ & 1 & 1 & 0 \\ 
\hline 
$x_3$ & 1 & 1 & 1 \\ 
\end{tabular} \\
$\mu_{R_2}(x_i,x_j)$=\begin{tabular}{c|c|c|c}
 & $x_1$ & $x_2$ & $x_3$ \\ 
\hline 
$x_1$ & 1 & 0 & 1 \\ 
\hline
$x_2$ & 1 & 1 & 1 \\ 
\hline 
$x_3$ & 1 & 0 & 1 \\ 
\end{tabular} \\
$\mu_{R_3}(x_i,x_j)$=\begin{tabular}{c|c|c|c}
 & $x_1$ & $x_2$ & $x_3$ \\ 
\hline 
$x_1$ & 1 & 0 & 0 \\ 
\hline 
$x_2$ & 1 & 1 & 1 \\ 
\hline 
$x_3$ & 1 & 1 & 1 \\ 
\end{tabular} \\
$\mu_{R_4}(x_i,x_j)$=\begin{tabular}{c|c|c|c}
 & $x_1$ & $x_2$ & $x_3$ \\ 
\hline 
$x_1$ & 1 & 1 & 1 \\ 
\hline 
$x_2$ & 0 & 1 & 1 \\ 
\hline 
$x_3$ & 0 & 1 & 1 \\ 
\end{tabular} \\
Визначаємо першу згортку:\\
$\mu_{Q_1}(x_i,x_j)$=\begin{tabular}{c|c|c|c}
 & $x_1$ & $x_2$ & $x_3$ \\ 
\hline 
$x_1$ & 1 & 0 &  0\\ 
\hline 
$x_2$ &  0& 1 & 0 \\ 
\hline 
$x_3$ &  0& 0 & 1 \\ 
\end{tabular} \\
Будуємо відношення строгої приналежності та недомінуючість:\\
$\mu_{Q_1^s}(x_i,x_j)$=\begin{tabular}{c|c|c|c}
 & $x_1$ & $x_2$ & $x_3$ \\ 
\hline 
$x_1$ & 0 &0  &0  \\ 
\hline 
$x_2$ &0  & 0 &0  \\ 
\hline 
$x_3$ &0  &0  & 0 \\ 
\hline
$\mu_{Q_1^n}(x)$ & 1 & 1 &1 \\
\end{tabular} \\
Визначаємо другу згортку:\\
$\mu_{Q_2}(x_i,x_j)$=\begin{tabular}{c|c|c|c}
 & $x_1$ & $x_2$ & $x_3$ \\ 
\hline 
$x_1$ & 1 & 0.15  &0.4  \\ 
\hline 
$x_2$ & 0.85 & 1 & 0.6  \\ 
\hline 
$x_3$ & 0.85 & 0.75 & 1 \\ 
\end{tabular} \\
Будуємо відношення строгої приналежності та недомінуючість:\\
$\mu_{Q_2^S}(x_i,x_j)$=\begin{tabular}{c|c|c|c}
 & $x_1$ & $x_2$ & $x_3$ \\ 
\hline 
$x_1$ & 0 & 0  &0  \\ 
\hline 
$x_2$ & 0.7 & 0 & 0  \\ 
\hline 
$x_3$ & 0.45 & 0.15 & 0 \\ 
\hline
$\mu_{Q_2^n}(x)$ & 0.3 & 0.85 & 1 \\
\hline 
$\mu_{Q_1^n}(x)$ & 1 & 1 &1 \\
\hline 
$\mu_{Q^n}(x)$ & 0.3 & 0.84 & 1
\end{tabular} \\
Отже, чітку не домінуюча альтернатива це $x_3$.
\end{exs}
\chapter{Дискретне програмування}
\section{Дискретне програмування:початок}
\begin{eqnarray}
&\max f(x_1,\ldots,x_n)\\
& g_i(x_1,\ldots,x_n)\leq b_i,\ifom\\
&x_j\geq0,\jfon\\
&x_j\in D_j
\end{eqnarray}
Типи задач:
\begin{itemize}
\item Задачі програмування в цілих числах;
\item Задачі булевого програмування;
\item Загальний випадок дискретності $D_j$ - скінченна множина.
\end{itemize}
\subsection{Особливості}
При розв’язку задач програмування в цілих числах в загальному випадку неможливо використати прийом заміни задач неперервним аналогом, а далі округлення до найближчого цілого.\\
Точні методи розв’язку:
\begin{itemize}
\item Метод площин, що відсікають;
\item Метод гілок та границь;
\item Метод аналізу та відкидання варіантів;
\item Методи динамічного програмування.
\end{itemize}
Наближені методи пошуку:
\begin{itemize}
\item Метод випадкового пошуку;
\item Метод вектора спаду;
\item Метод локальної оптимізації;
\item Модифікації точних методів.
\end{itemize}
\section{Метод площин,що відсікають}
\begin{eqnarray}
&\max \set{\sumjon c_jx_j}\label{eq:3:1}\\ 
& \sumjon \aij \xj \leq \bi,\ifom\\
& \xj\geq0\label{eq:3:2}\\
&\xj\in\mZ\label{eq:3:3}
\end{eqnarray}
Ідея методу: від дискретної задачі переходимо до неперервного аналогу. Тобто, розв’язуємо задачу \eqref{eq:3:1}-\eqref{eq:3:2}. Якщо отриманий розв’язок цілий, то це й буде шуканий розв’язок. В іншому випадку, до системи обмеження додаємо нове обмеження, яке не задовольняє знайдений оптимальний, але не цілий план, але задовольняє будь-який цілий. Знову розв’язуємо неперервGну задачу. Знаходимо оптимальний розв’язок і перевіряємо, чи виконується умова \eqref{eq:3:3}. Далі аналогічно. Обмеження, що вводяться таким чином називаються \textbf{правильнім відсіканням}. Якщо правильне відсікання працює ефективно, то за скінченну кількість ітерацій ми прийдемо до оптимального розв’язку або до ознаки нерозв’язності.\\
\subsection{Умова збіжності}
Якщо цільова функція на множині розв’язків обмежена цілком (зверху і знизу), а множина оптимальних розв’язків є випуклим багатогранником і рядок, що спрямовує формується за симплекс-таблицею з не цілою компонентою, а рядок, що спрямовує формується по рядку симплекс-таблиці за умовою $\Theta_j = \min \left| \cfrac{\Delta_l}{x_{il}}\right|_{x_{il}<0}$, то алгоритм таки працює.