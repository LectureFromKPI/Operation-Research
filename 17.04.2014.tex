\section{Метод можливих напрямів} \marginpar{\framebox{17.04.2014}}
\subsection{Метод можливих напрямів з лінійними обмеженням}
\begin{eqnarray}
&\min f(x) \label{tr:8:1}\\
&A\cdot X \leq B \\
&H\cdot X=U\label{tr:8:2}
\end{eqnarray}
Нехай $x_1$ допустима точка задачі \eqref{tr:8:1}-\eqref{tr:8:2} для якої:
\begin{equation}
\system{A_1 x_1 = B_1\\ A_2 x_1< B_2}
\end{equation}
$s$ - можливий напрямок в точці $x_1$, якщо виконується умова:
\begin{eqnarray}
A_1 S \leq 0 \label{tr:8:3}\\
H S = 0 \label{tr:8:4}
\end{eqnarray}
Вектор $s$ можливий напрямок спуску в точці $x_1$, якщо виконуються умови\eqref{tr:8:3},\eqref{tr:8:4} і при цьому:
\begin{equation}
\triangledown \mt{f}\cb{x_1} s < 0
\end{equation}
Для того, щоб знайти такий вектор, необхідно мінімізувати $\min \triangledown \mt{f}(x_1)$. Така умова називається \textbf{умовою нормування}. В результаті, отримуємо наступну задачу:
\begin{eqnarray}
&\triangledown \mt{f}\cb{X_1} s < 0 \\
&A_1 S \leq 0 \\
&H S = 0\\
&-1 \leq S_j \leq 1
\end{eqnarray}
Оптимальний розв’язок даної задачі позначимо як $s_1$. Якщо значення цільової функції дорівнює нулю, то поточна точка оптимальний розв’язок задачі. Тобто, точка Куно-Такера. В іншому випадку все сумно.
\subsection{Алгоритм}
\paragraph{Початковий етап} Нехай задана задача \eqref{tr:8:1}-\eqref{tr:8:2}. Шукаємо допустиму точку $x_1$ і йдемо на першу ітерацію.
Нехай проведено $k-1$ ітерація, знайдена точка $x_k$, для якої 
\begin{equation*}
\system{
A_1 x_k = B_1\\
A_2 x_k < B_2
}
\end{equation*}
\paragraph{Перший етап} Складаємо та розв’язуємо задачу 
\begin{eqnarray}
&\min \triangledown \mt{f} (x_k) s \\
&A_1 S \leq 0 \\
&H S = 0 \\
&-1 \leq s_j \leq 1
\end{eqnarray}
І знайшли оптимальний розв’язок $s_k$. Якщо значення функції менше нуля, то $s_k$ можливий напрямок спуску і ми йдемо на другий етап. А в іншому випадку $x_k$ оптимальний розв’язок задачі.
\paragraph{Другий етап} Складаємо та розв’язуємо наступну задачу 
\begin{eqnarray}
&\min f(x_k+\la s_k)\\
&0\leq \la \leq \la_{\max}
\end{eqnarray}
Де права частина у обмеження, це максимальне значення $\la$ при якому точка $x_k+\la s_k$ залишається допустимою. Далі покладемо $x_{k+1} = x_k+\la_k s_k$ і йдемо на наступну ітерацію.
\begin{exs}
\begin{eqnarray}
&\min 2x_1^2 + 2x_2^2 - 2x_1x_2 -4x_1 -6 x_2\\
&x_1+x_2 \leq 2\\
&x_1 +5x_2\leq 5\\
&-x_1 \leq 0\\
&-x_2 \leq 0
\end{eqnarray}
\begin{equation}
\triangledown f(x) = \vect{4x_1 -2x_2 +4\\ 4x_2-2x_1-6}
\end{equation}
\begin{eqnarray}
&\min -4s_1 - 6s_2 \\
&-s_1\leq 0\\
&-s_2 \leq 0\\
&-1\leq s_1,s_2 \leq 1
\end{eqnarray}
$s_1=s_2=1;F=-10$\\
Ідемо на другий етап:
\begin{equation}
\min 2\cb{\la s_1}^2 + 2\cb{\la s_2}^2 -2 \la^2 s_1 s_2 -4\la s_1 - 6 \la s_2 
\end{equation}
Спростимо до нормального виду
\begin{eqnarray}
&\min 2\la^2 -10\la \\
&4\la -5 = 0\\
&\la =2.5
\end{eqnarray}
Також є обмеження:
\begin{equation}
\system{2\la \leq 2\\ 6\la \leq 5} \Rightarrow \la = \dfrac56
\end{equation}
Отже, $\la=\dfrac56$
\end{exs}
\subsection{Метод можливих напрямів для задач з нелінійними обмеженнями}
\begin{eqnarray}
&\min f(x) \label{tr:8:5}\\
&g_i(x) \leq 0,\ifom\label{tr:8:6}
\end{eqnarray}
Нехай $x$ допустима точка задачі \eqref{tr:8:5}-\eqref{tr:8:6}. Через множину $I = \set{i:g_i(x)=0}$ - множину індексів активних обмежень. Нехай, якщо $f(x)$ і $g_i(x),i\in I$ - диференційовані, а $g_i(x),i\not\in I$ - неперервні. Вектор $s$ можливий напрям спуску в точці $x$, якщо 
\begin{eqnarray}
&\triangledown \mt{f}(x) s <0\\
&\triangledown \mt{g_i} (x) s <0,i\in I
\end{eqnarray}
Для того, щоб знайти цей вектор, потрібно $\min\set{\max\set{\triangledown \mt{f}(x)s,\triangledown \mt{g_i} (x) s}}$. Позначимо $\max\set{\triangledown \mt{f}(x)s,\triangledown \mt{g_i} (x) s} = z$. В результаті отримуємо наступну задачу
\begin{eqnarray}
&\min z\\
&\triangledown \mt{f}(x) s - z \leq 0\\
&\triangledown \mt{g_i}(x)s - z\leq 0,i\in I\\
&-1\leq s_j \leq 1
\end{eqnarray}
Позначимо оптимальний розв’язок данної задачі як $s_0,z_0$. Якщо $z_0=0$, то поточна точка є оптимальним розв’язком.
\subsubsection*{Алгоритм}
\paragraph{Початковий етап} Знаходимо допустиму точку і йдемо на першу ітерацію.\\
Нехай проведена $k-1$ ітерація я знайдена точка $x_k$. Позначимо через $I=\set{i:g_i(x_k)=0}$ - множину індексів активних обмежень. 
\paragraph{Перший етап} Складаємо та розв’язуємо наступну задачу:
\begin{eqnarray}
&\min z\\
&\triangledown \mt{f}(x) s - z \leq 0\\
&\triangledown \mt{g_i}(x)s - z\leq 0,i\in I\\
&-1\leq s_j \leq 1
\end{eqnarray}
Оптимальний розв’язок позначили $s_k,z_k$. Якщо $z_k=0$, то $x_k$ оптимальний розв’язок задачі. Інакше йдемо на другий етап
\paragraph{Другий етап} Складаємо та розв’язуємо наступну задачу 
\begin{eqnarray}
&\min f(x_k+\la s_k)\\
&0\leq \la \leq \la_{\max}
\end{eqnarray}
Де права частина у обмеження, це максимальне значення $\la$ при якому точка $x_k+\la s_k$ залишається допустимою. Далі покладемо $x_{k+1} = x_k+\la_k s_k$ і йдемо на наступну ітерацію.
\begin{exs}
\begin{eqnarray}
&\min \cb{x_1-2x_2}^2 + 5\cb{x_1-8}^2\\
&x_1^2 - 5 x_2 \leq 10\\
&x_1 + 2x_2 \leq 8
\end{eqnarray}
\begin{equation}
\triangledown f(x) = \vect{2(x_1-2x_2) + 10 \cb{x_1-8}\\ -4\cb{x_-2x_2}}
\end{equation}
\begin{equation}
\triangledown f(x') = \vect{-80\\0}
\end{equation}
Складемо задачу:
\begin{eqnarray}
&\min z \\
&-80 s_1 - z \leq 0\\
&-1 \leq s_1 \leq 1
\end{eqnarray}
$s_1 = 1,z=-80$\\
\begin{eqnarray}
&\min \set{\cb{\la s_1}^2 + 5 \cb{\la s_1-8}^2}\\
&2\la + 10\cb{\la - 8}=0\\
& \la = \dfrac{20}3
\end{eqnarray}
Розглянемо обмеження:
\begin{equation}
\system{\la^2 \leq 10 \\ \la\leq 8} \Rightarrow \la = \sqrt{10}
\end{equation}
Запишемо нову точку:
\begin{equation}
x = \vect{\sqrt{10}\\0}
\end{equation}
\end{exs}
\chapter{Геометричне програмування}
\section{Прості задачі геометричного програмування}
Прості задачі - задачі без обмежень.
\begin{eqnarray}
g(t) = \sumion U_i(t) = \sumion c_i t_1^{a_{i1}} \cdot \ldots\cdot t_m^{a_{im}}\label{tr:8:7}
\end{eqnarray}
Для розв’язку таких задач використаємо нерівності о середньому геометричному та середньому арифметичному.\\
\begin{eqnarray}
&\cfrac1n \sumion U_i \leq \prod\limits_{i=1}^n U_i^{1/n}\\
&\cfrac1n \sumion U_i \delta_i \geq \prod\limits_{i=1}^n U_i^{\delta_i}\\
&\delta_i\geq 0\label{tr:8:8}\\
&\sumion \delta_i = 1\label{tr:8:9}\\
&U_i\delta_i = u_i \\
&u_i(t) = c_it_1^{a_{i1}}\cdot\ldots\cdot t_m^{a_{im}}\\
&\sumion u_i(t) \geq \prod\limits_{i=1}^n \cb{\cfrac{u_i}{\delta_i}}^{\delta_i} = \prod\limits_{i=1}^n \cb{\cfrac{c_i}{\delta_i}} t_1^{D_1}\ldots t_m^{D_m}\\
&D_j = \sumion a_{ij} \delta_i\\
\end{eqnarray}
Підберемо вагові коефіцієнти таким чином, щоб:
\begin{equation}\label{tr:8:10}
\sumion \aij \delta_i = 0
\end{equation}
І підставимо:
\begin{equation}
g(t) \leq V(\delta)
\end{equation}
$\delta_i$ - це змінні, для яких виконуються умови нормалізації, ортогоналізації та невід’ємності, тобто умови \eqref{tr:8:8},\eqref{tr:8:9},\eqref{tr:8:10}
\begin{teor}
Нехай $t^*$ точка мінімума функції \eqref{tr:8:7}, тоді існує $\delta^*$ для якої виконуються умови \eqref{tr:8:8},\eqref{tr:8:9},\eqref{tr:8:10} і в цій точці значення двоїстої функції співпадає зі значенням прямої функції $V(\delta^*) = g(t^*)$
\end{teor}
\begin{proof}
\begin{eqnarray}
&\dd {g(t^\ast)}{t_j} = 0\\
&\dd {g(t^\ast)}{t_j}\cdot t_j^\ast = \sumion \dd{u_i(t^\ast)}{t_j} \cdot t_j^\ast = \sumion u_i(t^\ast) \aij = 0
\end{eqnarray}
Так як $g(t^\ast)\neq0$:
\begin{eqnarray}
&\sumion \cfrac{u_(t^\ast)}{g(t^\ast)} \aij = 0\\
&\cfrac{u_(t^\ast)}{g(t^\ast)} = \delta_i^\ast\\
&\sumion \delta_i^\ast \aij = 0
\end{eqnarray}
Отже, умова ортогональності виконується\\
Перевіримо умову нормалізації:\\
\begin{equation}
\sumion \delta_i^\ast = \sumion \cfrac{u_(t^\ast)}{g(t^\ast)} = 1
\end{equation}
\begin{eqnarray}
g(t^\ast) = \prod\limits_{i=1}^n g(t^\ast)^{\delta_i^\ast} = \prod\limits_{i=1}^n \cb{\cfrac{u_i(t^\ast)}{\delta_i^\ast}}^{\delta_i^\ast} = v(\delta^\ast)
\end{eqnarray}
\end{proof}