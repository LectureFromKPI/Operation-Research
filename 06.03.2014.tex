Виведемо формулу для правильного відсікання. \marginpar{\framebox{06.03.2014}}\\
Нехай всі обмеження задачі задані в канонічному виді
\begin{equation}\label{tr:4:1}
Ax = B
\end{equation}
Розмірність матриці $A=(m\times n)$. Ранг матриці $A$ дорівнює $m$. Нехай $m<n$. Базис містить $m$ векторів. Тоді $m$ стовпчиків матриці $A$ утворює базисну матрицю. Інші стовпчики утворюють небазисну матрицю. Таким же чином розіб’ємо вектор $x$ на базисний та небазисний і запишемо систему \eqref{tr:4:1} в наступному вигляді
\begin{equation}\label{tr:4:2}
A_bx_b + A_{nb}x_{nb} = B
\end{equation}
Помножимо систему \eqref{tr:4:2} на матрицю, обернену до базисної
\begin{equation}\label{tr:4:3}
A^{-1}_bA_bx_b + A^{-1}_bA_{nb}x_{nb} = A^{-1}_bB
\end{equation}
І з системи \eqref{tr:4:3} запишемо розв’язок системи \eqref{tr:4:1}
\begin{equation}
x_b = A^{-1}_bB - A^{-1}_bA_{nb}x_{nb}
\end{equation}
\begin{equation}\label{tr:4:6}
x_i = a_{i0} - \suml_{j\in I_{nb}} \aij\xj
\end{equation}
Позначимо через $\gamma_{i0}$ дробову частину $a_{i0}$
\begin{equation}\label{tr:4:4}
a_{io} = \bb{a_{i0}}+\gam_{i0}
\end{equation}
А через $\gamij$ дробову частину \aij
\begin{equation}\label{tr:4:5}
\aij = \bb{\aij} + \gamij
\end{equation}
Підставим \eqref{tr:4:4}-\eqref{tr:4:5} у \eqref{tr:4:6}
\begin{equation}
x_i -\bb{a_{i0}}+\suml_{j\in I_{nb}} \bb{\aij}\xj = \gam_{i0} - \suml_{j\in I_{nb}} \gamij\xj
\end{equation}
Нехай $x_i$ ціле. Позначимо $\gam_{i0} - \suml_{j\in I_{nb}} \gamij\xj = \xi$ - також буде цілою.\\
Можливі два випадки:
\begin{itemize}
\item $\xi\geq 1$;
\item $\xi\leq 0$.
\end{itemize}
Розглянемо перший випадок $\cb{\xi\geq 1}$:
\begin{eqnarray}
&\gam_{i0} - \suml_{j\in I_{nb}} \gamij\xj \geq 1\\
&\Rightarrow \gam_{i0} \geq 1 \suml_{j\in I_{nb}} \gamij\xj \geq 1
\end{eqnarray}
Це протиріччя з умовою, що $\gam_{i0}$ - дробова частина. Отже, можливий лише один варіант $\cb{\xi\leq 0}$:
\begin{eqnarray}
\gam_{i0} - \suml_{j\in I_{nb}} \gamij\xj \leq 0\\
\gam_{i0} \leq \suml_{j\in I_{nb}} \gamij\xj
\end{eqnarray}
Не цілий план цій умові не задовольняє, а цілій задовольняє. Отримали правильно формулу, запишемо її:
\begin{equation}\label{tr:4:7}
 -\suml_{j\in I_{nb}} \gamij\xj \leq -\gam_{i0}
\end{equation}
Приводимо обмеження \eqref{tr:4:7} до канонічного виду
\begin{equation}
-\suml_{j\in I_{nb}} \gamij\xj +x_{n+1} =  -\gam_{i0}
\end{equation}
Цей рядом дописуємо у симплекс-таблицю оптимального розв’язку неперервної задачі. І далі розв’язуємо задачу двоїстим симплекс-методом. $n+1$ рядок буде спрямовуючим. Сукупність ітерацій двоїстим симплекс-методом починаючи від цього розв’язку та до нового оптимального називається \textbf{великою ітерацією}.\\
Можливі наступні результати великої ітерації:
\begin{description}
\item[Ознака оптимальності:] $\forall i:x_i=x_{i0}\geq 0,x_i\in\mZ$;
\item[Ознака формування нового відсікання:] $\forall i:x_i=x_{i0}\geq 0,\exists i^*:x_{i^*}\not\in\mZ$;
\item[Ознака неров’язності:] $\exists i^*:x_{i^*} = x_{i^*0}<0,\xij\geq 0$.
\end{description}
Змінні $x_{n+1},x_{n+2},\ldots$ називається \textbf{додатковими}, вони виводяться з базису на першій ітерації. Якщо якусь додаткову змінну знову необхідно ввести в базис, то відповідний рядок та стовпчик можна викинути. Дробові частини завжди записуються зі знаком мінус.
\section{Метод гілок та границь для задач програмування в цілих числах}
Записуємо математичну модель
\begin{eqnarray}
&\max\sumjon \cj\xj\label{tr:4:8}\\
&\sumjon \aij\xj\leq \bi,\ifom\\
&\xj\geq0,\jfon\label{tr:4:9}\\
&\xj\in\mZ,\jfon
\end{eqnarray}
\subsection*{Перший етап}
Відкидаємо умову того, що $\xj\in\mZ$ і розв’язуємо неперервну задачу \eqref{tr:4:8}-\eqref{tr:4:9}. Оптимальний розв’язок позначимо як $x_0$. Якщо даний розв’язок цілий, то воно і буде шуканим. В іншому випадку визначаємо верхню межу (оцінку) для шуканого розв’язку. Оценку будемо позначати буквою $\xi\cb{G_0}=f\cb{x_0}$. І йдемо на першу ітерацію.
\subsection*{Ітерації}
Нехай проведено $k$-ітерацій. Є ряд підмножин $\set{G_1^k,\ldots,G_v^k}$. Для кожної підмножини пораховані оцінки $\set{\xi\cb{G_1^k},\ldots,\xi\cb{G_v^k}}$. $k+1$-ітерація.
\subsubsection{Перший етап: ветвление}
Для ще одного ветвления обираємо підмножину з максимальною оцінкою. Нехай це підмножина $G_i^k:\xi\cb{G_i^k}=\max\set{\xi\cb{G_j^k}}$. $G_j^k$ - множина вершин, що вісять. Тобто ті підмножини, які ще не ветвлились. Розбиваємо підмножину $G_i^k$ на дві множини, що не перетинаються: $G_i^k = G_1^{k+1}\cup G_2^{k+1}$. Для цього в плані $x_i^k$ обираємо деяку не цілу компоненту. Нехай це $x_r = x_{r0}$. Першу множину отримуємо таким чином:
\begin{equation}
G_1^{k+1} = G_i^K \cap \set{x_r:x_r\leq\bb{x_{r0}}}
\end{equation}
Другу множину отримуємо за такою формулою:
\begin{equation}
G_2^{k+1} = G_i^K \cap \set{x_r:x_r\geq\bb{x_{r0}}+1}
\end{equation}
\subsubsection{Другий етап}
Знаходимо оптимальні розв’язки на підмножинах і обчислюємо оцінки для підмножин.\\
В симплекс-таблицю оптимального розв’язку на підмножині $G_i^k$ дописуємо обмеження
\begin{equation}
x_r \leq \bb{x_{ro}}
\end{equation}
Розв’язуємо цю задачу двоїстим симплекс-методом і оптимальний план на підмножині $G_1^{k+1}$ позначимо $x_1^{k+1}$. І оцінку $\xi\cb{G_1^{k+1}} = f\cb{x_1^{k+1}}$.\\
В симплекс-таблицю оптимального розв’язку на підмножині $G_i^k$ дописуємо обмеження:
\begin{equation}
x_r \geq \bb{x_{ro}} +1
\end{equation}
Розв’язуємо цю задачу двоїстим симплекс-методом і оптимальний план на підмножині $G_2^{k+1}$ позначимо $x_2^{k+1}$. І оцінку $\xi\cb{G_2^{k+1}} = f\cb{x_2^{k+1}}$.
\subsubsection{Третій етап: перевірка ознаки оптимальності}
Якщо план $x_1^{k+1}$ цілий і оцінка для відповідної підмножини $\xi\cb{G_1^{k+1}} \geq \max\xi\cb{G_j^k}$, то план $x_1^{k+1}$ і буде оптимальним планом. У іншому випадку на наступну ітерацію.
\subsubsection{Особливості}
\begin{itemize}
\item Обмеження, що вводяться грають роль відсікань;
\item Якщо на деякій підмножині $G_i=\emptyset$ задача нерозв’язна, то оцінка для відповідної підмножини $\xi\cb{G_i}=+\infty$, якщо розв’язується задача мінімізації та  $\xi\cb{G_i}=-\infty$, якщо розв'язується задача максимізації;
\item Якщо всі коефіцієнти цільової функції цілі, то оцінку можна визначити за такою формулою $\xi\cb{G_0} = \bb{f\cb{x_0}}+1$;
\item Якщо розв’язується задача мінімізації, то для ще одного ветвления обирається підмножина з мінімальною оцінкою, а ознакою оптимальності буде те, що план є цілим та $\xi\cb{G_1^{k+1}} \leq \min\xi\cb{G_j^k}$.
\end{itemize}
\begin{exs}
\begin{eqnarray}
&\max\cb{2x_1+3x_2}\\
&x_1+4x_2\leq 14\\
&2x_1+3x_3\leq 12\\
&x_1,x_1\geq 0\\
&x_1,x_2\in\mZ
\end{eqnarray}
Приводимо до канонічного виду та позбавляємося від умови цілих чисел.
\begin{eqnarray}
&\max\cb{2x_1+3x_2}\\
&x_1+4x_2+x_3= 14\\
&2x_1+3x_3+x_4= 12\\
&x_1,x_1\geq 0
\end{eqnarray}
Далі розв’язуємо звичайним симплекс методом. Оскільки отриманий результат не цілий, переходимо до гілляння.
Отримуємо таблицю:\\
\begin{tabular}{c|c|c|c|c|c|c|c}
\hline
&&&2 & 3 & 0 & 0\\
\hline
& $B_x$ & $A_0$ & $A_1$ & $A_2$ & $A_3$ & $A_4$ & $A_5$ \\
\hline
3 & $x_2$ & $\cfrac{16}5$ & 0 & 1 & $\cfrac25$ & $-\cfrac15$ & 0\\
\hline
2 & $x_1$ & $\cfrac65$ & 1 & 0 & $-\cfrac35$ & $\cfrac45$ & 0\\
\hline
& $\Delta$ & 12 & 0 & 0 & 0 & 1\\
\hline
$\leftarrow$&  $x_5$ & $-\cfrac15$ & 0 & 0 & $-\cfrac25$ & $-\cfrac45$ &1 \\
\hline
&&&&&$\uparrow$&\\
\end{tabular}\\
Будуємо нову симплекс-таблицю:\\
\begin{tabular}{c|c|c|c|c|c|c|c|c}
&&&2 & 3 & 0 & 0\\
\hline
& $B_x$ & $A_0$ & $A_1$ & $A_2$ & $A_3$ & $A_4$ & $A_5$ & $A_6$\\
\hline
3 & $x_2$ & 3 & 0 & 1 & 0 & -1 & 1 & 0\\
\hline
2 & $x_1$ & $\cfrac32$ & 1 & 0 & 0 & 2 & $-\cfrac32$ & 0\\
\hline
0&  $x_3$ & $\cfrac12$ & 0 & 0 & 1 & 2 & $-\cfrac52$ & 0\\
\hline
$\leftarrow$& $x_6$ & $-\cfrac12$ & 0 & 0 & 0  & 0 & $-\cfrac12$ & 1\\
&&&&&&&$\uparrow$&
\end{tabular}\\
І ще одну нову симплекс-таблицю:\\
\begin{tabular}{c|c|c|c|c|c|c|c}
&&&2 & 3 & 0 & 0\\
\hline
& $B_x$ & $A_0$ & $A_1$ & $A_2$ & $A_3$ & $A_4$ & $A_5$ \\
\hline
3 & $x_2$ & 2 & 0 & 1 & $\cfrac25$ & $-\cfrac15$ & 0\\
\hline
2 & $x_1$ & 3 & 1 & 0 & $-\cfrac35$ & $\cfrac45$ & 0\\
\hline
& $x_3$ & 3 &&&&&
\end{tabular}\\
Отримали розв’язок $f(x) = 12$
\end{exs}